\section{Progress and Difficulties} \label{sec:progress}

Until this report, we have achieved several milestones as well as met several difficulties in the development of the syntactic qualifier. Though it seems most proofs have been reconstructed with the syntactic qualifier, the last few lemmas are the most hard parts and require much efforts, so we might still have a long way to go on this topic.

The first remarkable progress is to modify almost all the basic lemmas related to qualifiers in the original system. The modification is necessary because the sub-qualifier relations of syntactic qualifier rely on the context while that of original set qualifier do not, so that we have to embed context into these judgements. Meanwhile, since the definition of opening and substituting is completely different from the set qualifier, so we adopt new proof trees for these lemmas.

As an important part in the type safety proof, we finish the weakening lemma of the \langstar system with syntactic qualifiers. For these lemmas, we replace the equality in the assumptions of set qualifiers with equivalence assumptions of syntactic qualifiers. We adopt the rewrite strategy for equivalent expressions instead of direct rewrite for equal expression to preserve the proof trees. As a side note, for these expressions not related to sub-qualifier relations, we can keep the equality because the operations besides sub-qualifier relations are normal functions.

We are currently stuck and working on the substitution lemma, with which we are able to finish the type safety theorem proof. There are three non-trivial cases in the substitution lemma: variable case, function abstraction case, and application case. We have completed the application case, almost finish the variable case with two minor goals related to context issue, and stuck on the function abstraction case. Under the current proof tree, we require the distributive law to finish the goal but the distributive law is not supported in our syntactic qualifier system (see discussion \ref{sec:distributive}). This difficulty reflects that our equivalence rewrite is not as strong as the equality rewrite in set qualifiers. Possible ways to solve this problem might be either: adopt a new proof tree to avoid this subcase, prove a fake ``distributive'' lemma in this special case instead of prove a general distributive law (it's the only case requires distributive law till now), or extend the equivalence to a stronger one (see Section \ref{sec:next-step}).   